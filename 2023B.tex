\documentclass[11pt,twoside,a4paper]{article}
\usepackage{amsmath,amsfonts}
\usepackage{algorithmic}
\usepackage{algorithm}
\usepackage{array}
\usepackage[caption=false,font=normalsize,labelfont=sf,textfont=sf]{subfig}
\usepackage{textcomp}
\usepackage{stfloats}
\usepackage{ctex}
\usepackage{url}
\usepackage{verbatim}
\usepackage{graphicx}
\usepackage{cite}
\usepackage{amsmath,amsfonts}
\usepackage{algorithmic}
\usepackage{algorithm}
\usepackage{array}
\usepackage[caption=false,font=normalsize,labelfont=sf,textfont=sf]{subfig}
\usepackage{textcomp}
\usepackage{stfloats}
\usepackage{url}
\usepackage{verbatim}
\usepackage{graphicx}
\usepackage{cite}
\usepackage{mathptmx}
\usepackage[justification=centering]{caption}
\usepackage[T1]{fontenc}
\usepackage{hyperref}
\usepackage{amssymb}
\usepackage{float}
\usepackage{subcaption}
\begin{document}
	\section{模型假设和约定}
	
	\section{符号说明}
	
	\section{基本模型建立}
\subsection{多波束测线二维平面模型}
	\begin{figure}[htbp]
		\centering
		\includegraphics[scale=1.5]{3}
		\caption{多波束探测的工作原理}
		\label{1}
	\end{figure}
	
	多波束探测系统在与航迹垂直的平面内发射多个波束,其工作原理如图\ref{1}所示,在海底平坦的海域,其能够测出以测线为轴线具有一定宽度的全覆盖水深条带,其覆盖宽度$W$随换能器开角$\theta$和水深$D$的变化而变化
	\begin{equation}
		W=2 \times D \tan{\frac{\theta}{2}} \text{。}
	\end{equation}	
	\begin{figure}[h]
		\centering
		\includegraphics[scale=0.4]{6}
		\caption{覆盖宽度、测线间距和重叠率之间的关系}
		\label{3}	
	\end{figure}

若两个测线相互平行,如图 \ref{3}所示,其重叠率$\eta$可表示为对于每个覆盖宽度$W$的重叠宽度$W_\text{ 重}$与覆盖宽度总和的比值,即
	\begin{equation}
	\eta=\frac{W_\text{重} \times 2}{W+W}
	\label{eq2}
    \end{equation}
其中 $W_\text{重}=2\times\frac{W}{2}-d$,
经化简得
	\begin{equation}
	\eta=1-\frac{d}{W}
   \end{equation}



\begin{figure}[h]
	\centering
	\includegraphics[scale=0.9]{4}
	\caption{单测线斜坡中多波束探测的工作原理1}
	\label{2}
\end{figure}

 \begin{figure}[h]
	\centering
	\includegraphics[scale=0.9]{7}
	\caption{单测线斜坡中多波束探测的工作原理2}
	\label{5}
\end{figure}



	 对于坡度为$\alpha$的海底坡面,如图\ref{2}和图\ref{5} 所示,其覆盖宽度 $W$由侧线为轴线的左侧宽度$W_1$ 和右侧宽度$W_2$组成。对于左侧宽度$W_1$有
	
	 \begin{equation}
		\frac{D}{\sin \left(\frac{1}{2}\times \pi -\alpha-\frac{1}{2}\times \theta \right)}=\frac{W_1}{\sin( \frac{1}{2}\times \theta)} \text{。}
	 \end{equation}
	 经化简得
	 	 \begin{equation}
	  W_1=\frac{D \sin{(\frac{\theta}{2}})}{\cos{(\frac{\theta}{2}+\alpha})}
	 	\end{equation}
 	同理,对于右侧宽度$W_2$
 	 \begin{equation}
 		\frac{D}{\sin \left(\frac{1}{2}\times \pi +\alpha-\frac{1}{2}\times \theta \right)}=\frac{W_2}{\sin( \frac{1}{2}\times \theta)} \text{。}
 	\end{equation}
     经化简得
      \begin{equation}
     	W_2=\frac{D \sin{(\frac{\theta}{2}})}{\cos{(\alpha-\frac{\theta}{2}})}
     \end{equation}
    由此可得
     \begin{equation}
    	W=\frac{D \sin{(\frac{\theta}{2}})}{\cos{(\alpha-\frac{\theta}{2}})}+\frac{D \sin{(\frac{\theta}{2}})}{\cos{(\alpha+\frac{\theta}{2}})}
    	\label{eq1}
    \end{equation}
 
	
		\begin{figure}[h]
		\centering
		\includegraphics[scale=1]{5}
		\caption{双测线斜坡中多波束探测的工作原理}
		\label{4}
	\end{figure}
对于双测线工作的情况, 如图\ref{4}所示,根据公式(\ref{eq1}),
 \begin{equation}
	W_\text{测1}=\frac{D_1 \sin{(\frac{\theta}{2}})}{\cos{(\alpha-\frac{\theta}{2}})}+\frac{D_1 \sin{(\frac{\theta}{2}})}{\cos{(\alpha+\frac{\theta}{2}})}
\end{equation}	
 \begin{equation}
	W_\text{测2}=\frac{D_2 \sin{(\frac{\theta}{2}})}{\cos{(\alpha-\frac{\theta}{2}})}+\frac{D_2 \sin{(\frac{\theta}{2}})}{\cos{(\alpha+\frac{\theta}{2}})}
\end{equation}	
	$W_\text{测1}$ 和	$W_\text{测2}$ 之间的重叠面积$W_\text{重}$可以表示为
 \begin{equation}
 W_\text{重}=(1-\frac{\beta-d}{\beta})\frac{D}{sin{\alpha}}
\end{equation}	
其中$D$是海域中心处的海水深度,$\beta=\frac{D}{\tan{\alpha}}$ ,化简得
 \begin{equation}
	W_\text{重}=\frac{D}{\sin{\alpha}}+\frac{d}{\cos{\alpha}}-D
\end{equation}

	由公式(\ref{eq2})得	
	\begin{equation}
		\eta=\frac{W_\text{重} \times 2}{W_\text{测1}+W_\text{测2}}
	\end{equation}

\begin{figure}[h]
	\centering
	\includegraphics[scale=1]{9}
	\caption{多测线斜坡中多波束探测的工作原理}
	\label{7}
\end{figure}
对于多测线的情况为保证测量的便利性和数据的完整性,仅考虑相邻条带重叠率为$10\%$到$20\%$的情况,如图\ref{7}所示
其总重叠率$\eta_\text{总}$可表示为
\begin{equation}
	\eta_\text{总}=\sum_{i=1}^{n-1} \eta_i=\sum_{i=1}^{n-1} \frac{W_{\text {重 } i} \times 2}{W_{\text {测 } i}+W_{\text {测 } i+1}}
\end{equation}


\subsection{多波束测线三维立体模型}
在实际情况中,多波束测探系统将沿着航迹运动并进行多点连续测探,并最终获得真实的海底情况,如图\ref{6}所示。对于一个矩形待测海域,测线方向与海底坡面的法向在水平面上投影的夹角为$\beta$,其坡度为$\alpha$,考虑到本问题中坡度较小,在本模型中近似认为旋转不会对覆盖宽度造成影响,其随$\beta$的变化趋势如图所示,海域中心点处海水深度为$D$。以垂直于水平面方向为$Z$轴,以海底坡面的法向在水平面的投影为$X$轴,以坡面法向与水平面法向组成的平面法向方向为$Y$轴,建立如图\ref{6}所示的坐标系,证明详见附录一。
 \begin{figure}[h]
	\centering
	\includegraphics[scale=0.5]{12}
	\caption{坡度值随$\beta$变化曲线}
	\label{10}
\end{figure}



 \begin{figure}[h]
	\centering
	\includegraphics[scale=0.7]{8}
	\caption{多波束测线三维立体示意图}
	\label{6}
\end{figure}


对于沿测线方向每点探测的覆盖宽度 $W(x)$将其分解为$X$方向$W_X(x)$和$Y$方向$W_Y(x)$
我们将测线方向与$Y$轴所夹的锐角,记为$\theta_1$,	其可用$\beta$表示。

\begin{equation}
	\theta_1=\left\{\begin{array}{l}
	     \beta   \   \    \  \   \   0 \leq \beta \leq \frac{1}{2} \pi\\
	     \pi-\beta   \   \    \  \ \ \frac{1}{2} \pi \leq \beta \leq \pi\\
	     \beta-\pi  \   \    \  \  \pi \leq \beta \leq \frac{3}{2}\pi\\
	     2\pi-\beta   \   \    \   \ \  \frac{3}{2} \pi \leq \beta \leq 2\pi\\
		
	\end{array}\right.
\end{equation}

同时我们将相对于起始点的位移分解为$x_{x},x_{y}$.,由前面的分析可以得到,$x_{x}$会影响$W_Y{x}$的长度,
$x_{y}$会影响$W_X{x}$的长度

此时,其沿$Y$方向的位移分量$D_\text{水平}$可表示为
\begin{equation}
	D_\text{水平}=x |\cos \theta_1|
	\label{eq3}
\end{equation}



均可由公式\ref{eq1}计算得到

 \begin{equation}
	W_Y=\frac{D_0 \sin{(\frac{\theta}{2}})}{\cos{(\alpha-\frac{\theta}{2}})}+\frac{D_0 \sin{(\frac{\theta}{2}})}{\cos{(\alpha+\frac{\theta}{2}})}
	\label{eq4}
\end{equation}
其中$D_0$表示测线起始位置时的海水深度。
\begin{equation}
	W_X(x)=2 \times D(x)\times \tan{\frac{\theta_x}{2}}
\end{equation}


 其中$D(x)$是测线方向测量船距海域中心点处距离为$x$时的海水深度。

 \begin{figure}[h]
	\centering
	\includegraphics[scale=0.4]{10}
	\caption{$D_\text{水平}$图像}
	\label{8}
\end{figure}


\begin{figure}[h]
	\centering
	\includegraphics[scale=0.4]{11}
	\caption{$D(x)$图像}
	\label{9}
\end{figure}
根据公式\ref{eq3},$D(x)$可表示为
 \begin{equation}
	D(x)=D_0-D_\text{水平}\tan{\alpha}
	\label{eq5}
\end{equation}
$D(x)$和$D_\text{水平}$图像如图\ref{8},图\ref{9}所示。
联立公式\ref{eq4},\ref{eq5}得

	
\begin{equation}
	W_{\text {总 }}=\sqrt{W_Y^2+W_X^2(x)}
\end{equation}

\subsection{多波束测线三维 优化模型}
\subsubsection{多波束测线三维优化指标的提出}
依据已建立的多波束测线三维立体模型,可以得到沿测线方向每点探测的覆盖宽度$W(x)$, 则多波束测线覆盖面积$S(x)$可以表示为
\begin{equation}
	S(x)=\int W(x) d x
\end{equation}
所以在沿测线方向上单位距离扫描面积$S_x(x)$可以表示为
\begin{equation}
	S_x(x)=\frac{\int_a^b W(x) d x}{l_{ab}}
\end{equation}
\begin{figure}[h]
	\centering
	\includegraphics[scale=1]{13}
	\caption{海域二维示意图}
	
\end{figure}
\begin{figure}[h]
	\centering
	\includegraphics[scale=0.4]{14}
	\caption{海域三维示意图}
	
\end{figure}
为了确定初始位置的选择,在问题三中建立起以南北方向为$X$轴,东西方向为$Y$轴的二维坐标系
对于沿测线方向上单位距离扫描面积$S_x(x)$,我们先确定起始位置,接着在确定起始位置上的航迹方向上
\begin{equation}
	\begin{aligned}
		& l_1=\frac{\frac{1}{2} c-f}{\sin \theta_{11}} \\
		& l_2=\frac{\frac{1}{2} c+f}{\cos \theta_{12}}
	\end{aligned}
\end{equation}
其中$c$表示矩形海域的坡度变化边边长,$f$表示起始位置距离海域中心点处的距离,$\theta_{11}$表示沿坡度上升方向测迹航向,$\theta_{12}$表示沿坡度下降方向测迹航向










\begin{equation}
	\eta_{\text {三维 }}=\frac{\int W_{\text {重 }} d_x \times 2}{\int W_{\text {测 } 1} d_x+\int W_{\text {测 } 2} d_x}
\end{equation}

\begin{gather}
	\quad \min\sum_{i=1}^{n} x_{\text{南}i}+\sum_{i=1}^{n} x_{\text{北}i} \\
	\text { s.t. }\left\{\begin{array}{l}
		 i=\left\lceil\frac{l \sin \left(\frac{3}{2} \pi-\alpha-\beta\right)}{d}\right\rceil \\
		0 \leq \beta \leq 2 \pi\\
		    10\% \leq \eta \leq 20\%                    \\	
	\end{array}\right.
\end{gather}


\subsection{任意海域多波束测线的设计}

\begin{figure}[h]
	\centering
	\includegraphics[scale=0.4]{1}
	\caption{海域三维示意图}
	
\end{figure}

\begin{figure}[h]
	\centering
	\includegraphics[scale=0.2]{2}
	\caption{海域三维示意图}
	
\end{figure}
设计测线布局是一个复杂的任务,需要考虑多个因素,以满足给定的要求。
离散化海域:首先,将待测海域划分为一个离散的网格或栅格。确定网格的大小,以便能够容纳单波束测量的数据点。确定每个测线条带的宽度。这是一个关键步骤,它会直接影响到设计的测线布局。通常情况下,条带的宽度应足够小,以确保测量点之间的数据点不会漏掉,但又足够大,以减少测线的总长度。这个宽度可以根据单波束测量的分辨率和精度要求来确定。选择一个起始点,通常是待测海域的一个角或边界点。这个点将成为第一条测线的起点。开始设计测线布局,以确保覆盖整个待测海域,并满足重叠率要求。 





\end{document}